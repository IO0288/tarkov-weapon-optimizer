\documentclass[aspectratio=169]{beamer}

% Theme selection
\usetheme{Madrid}
\usecolortheme{whale}

% Packages
\usepackage{graphicx}
\usepackage{tikz}
\usetikzlibrary{shapes,arrows,positioning}
\usepackage{booktabs}
\usepackage{listings}
\usepackage{xcolor}
\usepackage[fontset=fandol]{ctex}

% Custom colors
\definecolor{codegreen}{rgb}{0,0.6,0}
\definecolor{codegray}{rgb}{0.5,0.5,0.5}
\definecolor{codepurple}{rgb}{0.58,0.82}
\definecolor{backcolour}{rgb}{0.95,0.95,0.92}

% Code listing style
\lstdefinestyle{mystyle}{
    backgroundcolor=\color{backcolour},
    commentstyle=\color{codegreen},
    keywordstyle=\color{magenta},
    numberstyle=\tiny\color{codegray},
    stringstyle=\color{codepurple},
    basicstyle=\ttfamily\footnotesize,
    breakatwhitespace=false,
    breaklines=true,
    captionpos=b,
    keepspaces=true,
    numbers=left,
    numbersep=5pt,
    showspaces=false,
    showstringspaces=false,
    showtabs=false,
    tabsize=2
}
\lstset{style=mystyle}

\begin{document}

% Title Info
\title[塔科夫改枪优化器]{塔科夫改枪优化器}
\subtitle{基于约束规划的自动化配置方案优化}
\author{Imposs1ble嗨}
\date{\today}

\begin{frame}
    \titlepage
\end{frame}

% Section 1: Introduction
\section{背景介绍}

% Slide 3: The Problem
\begin{frame}{问题背景:复杂的武器改装系统}
    《逃离塔科夫》拥有业界最复杂的武器改装系统之一。

    \vspace{0.3cm}
    \textbf{玩家面临的主要挑战:}
    \begin{itemize}
        \item \textbf{配件数量庞大:}超过2500种配件,涉及后坐力、人机工效、重量、精准度等多项属性
        \item \textbf{嵌套兼容结构:}配件可安装于其他配件之上(导轨 $\to$ 转接座 $\to$ 瞄具)
        \item \textbf{组合空间爆炸:}一把AR-15有50余个槽位,有效配置方案多达数万种
        \item \textbf{性价比难以评估:}高价配件是否物有所值?
        \item \textbf{购买限制:}价格实时波动,且受商人等级和跳蚤市场解锁条件限制
    \end{itemize}

    \vspace{0.3cm}
    \textbf{现状:}玩家需要相当的时间来熟悉各个配件的属性,或直接照搬网上的"版本答案",而不理解其背后的权衡逻辑或者盲目购买高价配件。
\end{frame}

% Slide 4: The Solution
\begin{frame}{解决方案:自动化配置优化}
    \textbf{项目目标:}
    在给定约束条件下,自动计算出数学意义上的最优改枪方案。

    \vspace{0.4cm}
    \begin{columns}
        \column{0.5\textwidth}
        \textbf{典型应用场景:}
        \begin{itemize}
            \item \textit{"预算20万卢布,如何改出最好的M4?"}
            \item \textit{"不限预算,AK的后坐力最低能达到多少?"}
            \item \textit{"当前商人等级下,能改出什么水平的武器?"}
            \item \textit{"增加10万预算,后坐力提升有多大?"}
            \item \textit{"不同预算档位下的最优改枪方案分别是什么?"}
        \end{itemize}

        \column{0.5\textwidth}
        \textbf{核心功能:}
        \begin{itemize}
            \item \textbf{多目标优化:}综合平衡后坐力、人机工效与价格
            \item \textbf{预算控制:}支持硬性上限或软性惩罚
            \item \textbf{商人等级过滤:}根据玩家等级筛选可购配件以及商人等级限制
            \item \textbf{自定义过滤器:}根据玩家需求,自定义过滤器,如必装槽位、必装配件、必装类别等,以及排除某些配件等
            \item \textbf{帕累托前沿:}忽略一个属性,可视化呈现其他属性之间的权衡曲线
            \item \textbf{枪匠任务:}自动求解枪匠任务,输出成本最优的改枪方案
        \end{itemize}
    \end{columns}
\end{frame}

% Section 2: Architecture
\section{系统架构}

% Slide: Pipeline Overview
\begin{frame}{整体流程}
    \begin{center}
    \begin{tikzpicture}[
        node distance=0.8cm,
        every node/.style={font=\small},
        box/.style={draw, rectangle, rounded corners, fill=blue!20, minimum width=2.4cm, minimum height=0.9cm, align=center},
        arrow/.style={->, thick, >=stealth, line width=1.5pt}
    ]
        % Main pipeline - horizontal
        \node[box] (fetch) {数据获取\\{\scriptsize Tarkov.dev API}};
        \node[box, right=of fetch] (compat) {构建兼容树\\{\scriptsize BFS遍历}};
        \node[box, right=of compat] (solve) {求解优化\\{\scriptsize CP-SAT}};
        \node[box, right=of solve, fill=green!25] (result) {最优方案};

        % User input from below
        \node[box, below=0.7cm of solve, fill=orange!25] (user) {用户约束};

        % Arrows
        \draw[arrow] (fetch) -- (compat);
        \draw[arrow] (compat) -- (solve);
        \draw[arrow] (solve) -- (result);
        \draw[arrow] (user) -- (solve);
    \end{tikzpicture}
    \end{center}
\end{frame}

% Slide 5: Fetching Data
\begin{frame}{数据获取}
    \begin{columns}
        \column{0.45\textwidth}
        \textbf{Tarkov.dev GraphQL API}

        \vspace{0.3cm}
        \begin{itemize}
            \item 获取全部武器、配件及兼容性规则
            \item \textbf{24小时内价格:}包含商人价格与跳蚤市场价格
            \item 商人等级要求与跳蚤市场解锁条件
            \item 本地缓存机制提升查询效率
        \end{itemize}

        \column{0.55\textwidth}
        \begin{center}
            \includegraphics[width=\textwidth]{imgs/tarkov_dev.jpg}
        \end{center}
    \end{columns}
\end{frame}

% Slide 6c: BFS Compatibility Tree
\begin{frame}{构建兼容性树}
    \begin{columns}
        \column{0.55\textwidth}
        \textbf{广度优先搜索(BFS)流程:}
        \begin{enumerate}
            \item 从基础武器的槽位出发
            \item 遍历每个槽位,获取所有可以安装在该槽位的配件
            \item 将这些配件加入待处理队列
            \item 继续探索每个配件\textit{自身}拥有的槽位
            \item 重复上述过程直至无新配件可以安装
        \end{enumerate}

        \column{0.45\textwidth}
        \begin{center}
        \begin{tikzpicture}[
            node distance=0.6cm,
            every node/.style={font=\scriptsize},
            slot/.style={draw, rectangle, rounded corners, fill=blue!15, minimum width=1.6cm, minimum height=0.4cm},
            item/.style={draw, rectangle, fill=green!15, minimum width=1.4cm, minimum height=0.4cm}
        ]
            \node[item] (gun) {M4A1};
            \node[slot, below left=0.5cm and -0.2cm of gun] (s1) {机匣槽};
            \node[slot, below right=0.5cm and -0.2cm of gun] (s2) {握把槽};
            \node[item, below=of s1] (upper) {上机匣};
            \node[slot, below=of upper] (s3) {枪管槽};
            \node[item, below=of s3] (barrel) {416mm枪管};

            \draw[->] (gun) -- (s1);
            \draw[->] (gun) -- (s2);
            \draw[->] (s1) -- (upper);
            \draw[->] (upper) -- (s3);
            \draw[->] (s3) -- (barrel);
        \end{tikzpicture}
        \end{center}
        \textit{\scriptsize 配件拥有槽位,槽位容纳配件}
    \end{columns}
\end{frame}

% Section 3: The Engine
\section{优化引擎}

% Slide 7: What is CP-SAT?
\begin{frame}{CP-SAT求解器简介}
    \textbf{CP-SAT} = 约束编程(Constraint Programming)+ 布尔可满足性(SAT)

    \vspace{0.3cm}
    \begin{block}{核心原理}
        Google OR-Tools库中的求解器,通过融合以下技术求解组合优化问题:
        \begin{itemize}
            \item \textbf{约束编程:}以声明式方式表达复杂规则
            \item \textbf{SAT求解:}高效的布尔逻辑推理引擎
            \item \textbf{线性规划:}优化数值目标函数
        \end{itemize}
    \end{block}

    \vspace{0.3cm}
    \textbf{为何适用于本问题?}
    \begin{itemize}
        \item 天然支持"选中/不选"的布尔决策建模
        \item 可处理数千个变量与约束条件
        \item 保证求得全局最优解,而非近似解
        \item 典型武器配置问题可在毫秒级完成求解
    \end{itemize}
\end{frame}

% % Slide 8: How CP-SAT Works
% \begin{frame}{CP-SAT工作原理}
%     \begin{columns}
%         \column{0.5\textwidth}
%         \textbf{1. 模型构建}
%         \begin{itemize}
%             \item 定义布尔变量:$X_i \in \{0, 1\}$
%             \item 添加约束条件
%             \item 设定目标函数
%         \end{itemize}

%         \vspace{0.3cm}
%         \textbf{2. 求解过程}
%         \begin{itemize}
%             \item 传播:推断确定值
%             \item 搜索:对未定变量进行分支
%             \item 回溯:遇到矛盾时返回
%             \item 剪枝:跳过不可行分支
%         \end{itemize}

%         \column{0.5\textwidth}
%         \textbf{3. 关键技术}
%         \begin{itemize}
%             \item \textit{单元传播:}约束确定的值立即应用
%             \item \textit{冲突学习:}记录失败原因以避免重复探索
%             \item \textit{界收紧:}利用目标值提前剪除次优分支
%         \end{itemize}

%         \vspace{0.3cm}
%         \textbf{效果:}高效遍历搜索空间,确保找到全局最优解。
%     \end{columns}
% \end{frame}

% Slide 9: Constraint Programming Overview
\begin{frame}{CP-SAT在武器配置中的应用}
    \textbf{核心思路:}明确定义合法配置的条件,由求解器自动搜索最优方案。

    \vspace{0.3cm}
    \begin{block}{模型三要素}
        \begin{enumerate}
            \item \textbf{决策变量:}为每个可用配件定义布尔变量$X_i$($X_i = 1$表示选中)
            \item \textbf{约束条件:}兼容性规则、预算限制、必装槽位等
            \item \textbf{目标函数:}最大化人机工效、最小化后坐力或价格
        \end{enumerate}
    \end{block}

    \vspace{0.3cm}
    求解器遍历数百万种配件组合,实时剪除不合法配置,最终返回数学意义上的最优方案。
\end{frame}

% Slide 8: Constraint 1 - Mutex
\begin{frame}{约束1:槽位互斥}
    \textbf{规则:}每个槽位最多安装一个配件。

    \vspace{0.3cm}
    \begin{block}{数学表达}
        对于槽位$s$及其候选配件集合$\{i_1, i_2, ..., i_n\}$:
        \[
        \sum_{i \in \text{slot}_s} X_i \le 1
        \]
        其中$X_i = 1$表示配件$i$被选中,$X_i = 0$表示未选中。
    \end{block}

    \vspace{0.3cm}
    \textbf{示例:}M4的握把槽可安装GRAL-S\textit{或}Ergo PSG-1,但二者不可同时安装。
\end{frame}

% Slide 9: Constraint 2 - Dependency
\begin{frame}{约束2:父子依赖}
    \textbf{规则:}子配件的安装必须以父配件已安装为前提。

    \vspace{0.3cm}
    \begin{block}{数学表达}
        若配件$c$需安装于配件$p$所拥有的槽位:
        \[
        X_c \le X_p
        \]
        父配件未选中时,子配件不可选中。
    \end{block}

    \vspace{0.3cm}
    \textbf{示例:}安装前握把的前提是护木已安装在武器上,速瞄镜的前提是镜架已安装。

    \vspace{0.2cm}
    \begin{center}
    \textit{武器} $\to$ \textit{护木} $\to$ \textit{前握把}
    \end{center}
\end{frame}

% Slide 10: Constraint 3 - Conflicts
\begin{frame}{约束3:配件冲突}
    \textbf{规则:}部分配件存在互斥关系,不可同时安装。

    \vspace{0.3cm}
    \begin{block}{数学表达}
        对于API定义的冲突配件对$(a, b)$:
        \[
        X_a + X_b \le 1
        \]
        二者最多选其一。
    \end{block}

    \vspace{0.3cm}
    \textbf{示例:}AK以及M4的榴弹发射器和一些护木存在冲突,游戏不允许同时安装。
\end{frame}

% Slide 11: Constraint 4 - Required Slots
\begin{frame}{约束4:必装槽位}
    \textbf{规则:}特定槽位\textit{必须}安装配件,武器才可以正常使用。

    \vspace{0.3cm}
    \begin{block}{数学表达}
        对于必装槽位$s$:
        \[
        \sum_{i \in \text{slot}_s} X_i \ge 1
        \]
        该槽位必须安装至少一个兼容配件。
    \end{block}

    \vspace{0.3cm}
    \textbf{AR-15平台必装槽位:}
    \begin{itemize}
        \item 枪管、导气座、握把、拉机柄、机匣、护木
    \end{itemize}

    \vspace{0.2cm}
    缺少任一配件,武器均无法在游戏中正常使用。
\end{frame}

% Slide 12: Budget Constraint
\begin{frame}{约束5:预算限制}
    \textbf{规则:}配置总成本不得超过玩家设定的预算。

    \vspace{0.3cm}
    \begin{block}{数学表达}
        \[
        \sum_{i} \text{Price}_i \cdot X_i \le \text{Budget}
        \]
        所有选中配件的价格之和必须在预算范围内。
    \end{block}

    \vspace{0.3cm}
    \textbf{价格软约束机制(BFS阶段剪枝):}
    \begin{itemize}
        \item 考虑商人等级(1-4级对应不同解锁价格)
        \item 对比跳蚤市场价格
        \item 自动选择每个配件的最低价购买渠道
    \end{itemize}
\end{frame}

% Slide 13: Objective Function
\begin{frame}{目标函数}
    \begin{block}{加权评分函数(最大化)}
        \[
        \text{Score} = W_E \cdot \text{人机} - W_R \cdot \text{后坐力} - W_P \cdot \text{价格}
        \]
        \vspace{-0.2cm}
        \begin{itemize}
            \item $W_E$:人机工效权重 \quad $W_R$:后坐力权重 \quad $W_P$:价格权重
        \end{itemize}
    \end{block}

    \vspace{0.2cm}
    \textbf{预设配置方案:}
    \begin{itemize}
        \item \textit{最高人机:}$W_E = 98\%, W_R = 1\%, W_P = 1\%$
        \item \textit{最低后坐力:}$W_E = 1\%, W_R = 98\%, W_P = 1\%$
        \item \textit{最低价格:}$W_E = 1\%, W_R = 1\%, W_P = 98\%$
        \item \textit{均衡方案:}$W_E = 34\%, W_R = 33\%, W_P = 33\%$
    \end{itemize}
\end{frame}

% Slide: UI - Optimizer
\begin{frame}{用户界面:优化配置}
    \begin{center}
        \includegraphics[width=0.9\textwidth]{imgs/page_optimize.png}
    \end{center}
\end{frame}

% Slide: Optimization Results - Recoil
\begin{frame}{优化结果:极限后坐力}
    \begin{center}
        \includegraphics[width=0.85\textwidth]{imgs/eg_recoil.png}
    \end{center}
\end{frame}

% Slide: Optimization Results - Ergo
\begin{frame}{优化结果:极限人机}
    \begin{center}
        \includegraphics[width=0.85\textwidth]{imgs/eg_ergo.png}
    \end{center}
\end{frame}

% Slide: Optimization Results - Balanced
\begin{frame}{优化结果:均衡方案}
    \begin{center}
        \includegraphics[width=0.85\textwidth]{imgs/eg_balanced.png}
    \end{center}
\end{frame}

% Slide 14: Stat Calculations
% \begin{frame}{属性计算规则}
%     \begin{columns}
%         \column{0.5\textwidth}
%         \textbf{人机工效}(加法叠加):
%         \[
%         \text{人机}_{\text{最终}} = \text{人机}_{\text{基础}} + \sum_{i} \text{人机}_i
%         \]
%         各配件的人机加成直接累加。

%         \vspace{0.5cm}
%         \textit{游戏内存在100点软上限。}

%         \column{0.5\textwidth}
%         \textbf{后坐力}(乘法叠加):
%         \[
%         \text{后坐力}_{\text{最终}} = \text{后坐力}_{\text{基础}} \times \left(1 + \sum_{i} r_i\right)
%         \]
%         其中$r_i$为百分比修正值(如$-0.05$表示$-5\%$)。

%         \vspace{0.3cm}
%         \textit{多个减后坐力配件叠加存在边际递减效应。}
%     \end{columns}
% \end{frame}

% Section 4: Features
\section{特色功能}

% Slide: Gunsmith Task Solver
\begin{frame}{枪匠任务求解器}
    \textbf{背景:}枪匠任务要求玩家按照指定规格改装武器,对新手玩家具有一定难度。

    \vspace{0.2cm}
    \begin{columns}
        \column{0.5\textwidth}
        \textbf{任务定义(JSON格式):}
        \begin{itemize}
            \item 指定武器型号
            \item 属性要求:
            \begin{itemize}
                \item 人机工效下限、后坐力上限
                \item 弹匣容量、重量、瞄准距离等
            \end{itemize}
            \item 必须安装的特定配件
            \item 必须包含的配件类别(如消音器)
        \end{itemize}

        \column{0.5\textwidth}
        \textbf{求解流程:}
        \begin{enumerate}
            \item 加载任务配置文件
            \item 将任务要求转化为约束:
            \item 目标函数:最小化总价格
            \item 输出成本最低的达标方案
        \end{enumerate}
    \end{columns}

    \vspace{0.2cm}
    \textbf{效果:}可自动求解全部28个枪匠任务,输出成本最优的通关方案。
\end{frame}

% Slide: Gunsmith List
\begin{frame}{枪匠任务列表}
    \begin{center}
        \includegraphics[width=0.9\textwidth,height=0.75\textheight,keepaspectratio]{imgs/eg_gunsmith.png}
    \end{center}
\end{frame}

% Slide: Gunsmith Details
\begin{frame}{枪匠任务求解详情}
    \begin{center}
        \includegraphics[width=0.9\textwidth,height=0.75\textheight,keepaspectratio]{imgs/eg_gunsmith_2.png}
    \end{center}
\end{frame}

% Slide 9: Pareto Frontier
\begin{frame}{帕累托前沿分析}
    \textbf{概念说明:}
    \begin{itemize}
        \item 帕累托前沿展示了两个冲突目标(如后坐力与人机工效)之间的最优权衡曲线。
    \end{itemize}

    \begin{center}
    \begin{tikzpicture}[scale=0.85]
        \draw[->] (0,0) -- (6,0) node[right] {\small 后坐力};
        \draw[->] (0,0) -- (0,4) node[above] {\small 人机};

        \draw[thick, blue] (0.5, 1) to[out=45, in=180] (2.5, 2.8) to[out=0, in=135] (5, 3.3);

        \node[fill=red, circle, inner sep=2pt, label={[font=\scriptsize]left:低后座}] at (0.5, 1) {};
        \node[fill=green, circle, inner sep=2pt, label={[font=\scriptsize]below right:均衡}] at (2.5, 2.8) {};
        \node[fill=red, circle, inner sep=2pt, label={[font=\scriptsize]right:高人机}] at (5, 3.3) {};

        \node[font=\small] at (1.2, 3.5) {不可达};
        \node[font=\small] at (4.5, 1) {非最优};
    \end{tikzpicture}
    \end{center}

    \textbf{应用价值}:帮助玩家直观判断后坐力与人机工效之间的权衡,选择最适合自己的配置方案。
\end{frame}

% Slide: Pareto Visualization
\begin{frame}{帕累托前沿可视化}
    \begin{center}
        \includegraphics[width=0.9\textwidth,height=0.75\textheight,keepaspectratio]{imgs/page_pareto.png}
    \end{center}
\end{frame}

% Slide: Future Work
\begin{frame}{尚未实现的功能}
    \begin{itemize}
        \item \textbf{武器隐藏属性}:目前无法获取武器的隐藏属性,如后坐力角度等
        \item \textbf{商人以物易物}:目前无法获取商人以物易物的物品列表,以及商人以物易物的价格
    \end{itemize}
\end{frame}

% Section 5: Conclusion
\section{总结}

% Acknowledgements
\begin{frame}{致谢}
    \begin{center}
        \textbf{\Large 数据来源}

        \vspace{0.2cm}
        \includegraphics[height=0.9cm]{logos/tarkov-dev-logo.png}

        \vspace{0.1cm}
        社区开源GraphQL API

        \vspace{0.4cm}
        \textbf{\Large AI编程助手}

        \vspace{0.3cm}
        \begin{tabular}{c @{\hspace{1cm}} c @{\hspace{1cm}} c}
            \includegraphics[height=1.1cm]{logos/claude-color.png} &
            \includegraphics[height=1.1cm]{logos/gemini-color.png} &
            \includegraphics[height=1.1cm]{logos/minimax-color.png} \\[0.2cm]
            \textbf{Claude Code} & \textbf{Gemini CLI} & \textbf{MiniMax} \\[0.1cm]
            Anthropic & Google & MiniMax \\
        \end{tabular}

        \vspace{0.3cm}
        \textit{感谢以上AI助手在代码编写、调试及文档撰写方面提供的支持}
    \end{center}
\end{frame}

\end{document}
